\documentclass[12pt, letterpaper]{article}

\usepackage[utf8]{inputenc}
\usepackage{float}
\usepackage{systeme}
\usepackage{amsmath}
\usepackage{amssymb}
\usepackage{enumitem}
\usepackage{amsfonts}
\usepackage{amsthm}
\usepackage{graphicx}
\usepackage[colorinlistoftodos]{todonotes}
\usepackage{pifont}
\usepackage{mdframed,color}
\usepackage[letterpaper, left=3cm, right=3cm, top=3cm, bottom=3cm]{geometry}
\newcommand{\Z}{\mathbb{Z}}
\newcommand{\N}{\mathbb{N}}
\newcommand{\C}{\mathbb{C}}
\newcommand{\Q}{\mathbb{Q}}
\newcommand{\R}{\mathbb{R}}
\newcommand{\F}{\mathbb{F}}
\newtheoremstyle{statement}{3pt}{3pt}{}{}{\bfseries}{:}{.5em}{}

\theoremstyle{statement}
\newtheorem*{atmProp}{Proposition}

\theoremstyle{statement}
\newtheorem*{atmStat}{Statement}

\newenvironment{atmProof}{\noindent\ignorespaces\paragraph{Proof:}}{\hfill \ding{122}\par\noindent}

\newenvironment{Solution}{\noindent\ignorespaces\paragraph{Solution:}}{\hfill \ding{122}\par\noindent}

\newcount\arrowcount
\newcommand\arrows[1]{
        \global\arrowcount#1
        \ifnum\arrowcount>0
                \begin{matrix}
                \expandafter\nextarrow
        \fi
}

\newcommand\nextarrow[1]{
        \global\advance\arrowcount-1
        \ifx\relax#1\relax\else \xrightarrow{#1}\fi
        \ifnum\arrowcount=0
                \end{matrix}
        \else
                \\
                \expandafter\nextarrow
        \fi
}

\newcommand{\func}[2]{\operatorname{#1}(#2)}
\newcommand{\trans}[2]{\func{#1}{\Vec{#2}}}

\title{Mastery Homework 8}
\author{Rafael Laya}
\date{Fall 2018}

\begin{document}
    \maketitle

    \section*{Section 6.1}
    \subsection*{Problem 26}
    \begin{atmStat}
    Let $\Vec{u}=\begin{bmatrix} 5\\-6\\7 \end{bmatrix}$, and let $W$ be the set of all $\Vec{x} \in \R^3$ such that $\Vec{u} \cdot \Vec{x}=0$. What theorem in Chapter 4 can be used to show that $W$ is a subspace of $\R^3$? Describe $W$ in geometric language.
    \end{atmStat}
    \begin{Solution}
    The geometric place of all vectors perpendicular to the vector $\Vec{u}$ in $\R^3$ is the plane $P$ that is perpendicular to $\Vec{u}$ that goes through the origin. That is, the plane $5x-6y+7z=0$. A plane in $\R^3$ is spanned by two vectors. The span of any set of vectors in a vector space $V$ is a subspace of $V$ by theorem 1 in chapter 4. Therefore $W$ has to be subspace of $\R^3$. For instance, solve for $z$ in the equation for the plane and find two vectors in the plane that are not parallel: $z = \frac{-5x+6y}{7}$. When $x=y=1$ we have $z=\frac{-5+6}{7}=\frac{1}{7}$ and $x=2y=2$ we have $z=\frac{-10+6}{7}=-\frac{4}{7}$ And so $W = \operatorname{Span}\left(\left\{ \begin{bmatrix} 1\\1\\\frac{1}{7} \end{bmatrix}, \begin{bmatrix} 2\\1\\-\frac{4}{7} \end{bmatrix}\right\}\right)$.
    \end{Solution}
    
    \subsection*{Problem 28}
    \begin{atmStat}
    Suppose a vector $\Vec{y}$ is orthogonal to vectors $\Vec{u}$ and $\Vec{v}$. Show that $\Vec{y}$ is orthogonal to the vector $\Vec{u} + \Vec{v}$.
    \end{atmStat}
    \begin{Solution}
    Suppose $\Vec{y}$ is orthogonal to $\Vec{u}$ and $\Vec{v}$ in some vector space $V$ with an inner product. Then $\Vec{y}\cdot\Vec{u} = 0$ and $\Vec{y}\cdot\Vec{v}=0$. Now consider the following:
    
    \begin{align*}
    \Vec{y}\cdot(\Vec{u}+\Vec{v}) &= \Vec{y}\cdot\Vec{u} + \Vec{y}\cdot\Vec{v} \\
    &= 0 + 0 \\
    &= 0
    \end{align*}
    
    Therefore $\Vec{y}$ is orthogonal to the vector $\Vec{u}+\Vec{v}$.
    
    \end{Solution}
    
    \section*{6.2}
    \subsection*{Problem 32}
    \begin{atmStat}
    Let $\left\{ \Vec{v_1}, \Vec{v_2}\right\}$ be an orthogonal set of nonzero vectors, and let $c_1, c_2$ be any nonzero scalars. Show that $\left\{ c_1\Vec{v_2}, c_2\Vec{v_2}\right\}$ is also an orthogonal set. Since orthogonality of a set is defined in terms of pairs of vectors, this shows that if the vectors in an orthogonal set are normalized, the new set will still be orthogonal.
    \end{atmStat}
    \begin{Solution}
    Suppose $\left\{ \Vec{v_1}, \Vec{v_2} \right\}$ is an orthogonal set of nonzero vectors in a vector space $V$ with inner product. Let $c_1, c_2$ be any nonzero scalars. Then $\Vec{v_1}\cdot\Vec{v_2}=0$. Now consider the following:
    
    \begin{align*}
        (c_1\Vec{v_1})\cdot(c_2\Vec{v_2}) &= (c_1)(c_2)(\Vec{v_1}\cdot\Vec{v_2}) \\
        &= (c_1c_2)(0) \\
        &= 0 
    \end{align*}
    
    Then $c_1\Vec{v_1}$ is orthogonal to $c_2\Vec{v_2}$ and the set $\left\{ c_1\Vec{v_1},c_2\Vec{v_2}\right\}$ is an orthogonal set.
    \end{Solution}
    
    \subsection*{Problem 33}
    \begin{atmStat}
    Given $\Vec{u}\neq\Vec{0}$ in $\R^n$, let $L=\operatorname{Span}\left\{\Vec{u}\right\}$. Show that the mapping $\Vec{x}\mapsto\operatorname{proj_L}(\Vec{x})$ is a linear transformation.
    \end{atmStat}
    \begin{Solution}
    Let $\operatorname{T}:\R^n\longrightarrow\R^n$ be the transformation such that $\operatorname{T}(\Vec{x})=\operatorname{proj_L}(\Vec{x})$ for some $\Vec{u}\neq\Vec{0}$ in $\R^n$. Where $L=Span(\Vec{u})$. Since $L$ contains one vector, we find an equation for $\operatorname{T}$:
    
    $$
    \operatorname{T}(\Vec{x})=\frac{\Vec{x}\cdot\Vec{u}}{\Vec{u}\cdot\Vec{u}}\Vec{u}
    $$
    
    Now consider any two vectors $\Vec{x_1}, \Vec{x_2}\in\R^n$ and $c\in\R$. Then: 
    
    \begin{align*}
        \func{T}{\Vec{x_1}+\Vec{x_2}} &= \frac{(\Vec{x_1}+\Vec{x_2})\cdot\Vec{u}}{\Vec{u}\cdot\Vec{u}}\Vec{u} \\ 
        &= \frac{\Vec{x_1}\cdot\Vec{u}+\Vec{x_2}\cdot\Vec{u}}{\Vec{u}\cdot\Vec{u}}\Vec{u} \\
        &= \left(\frac{\Vec{x_1}\cdot\Vec{u}}{\Vec{u}\cdot\Vec{u}} + \frac{\Vec{x_2}\cdot\Vec{u}}{\Vec{u}\cdot\Vec{u}}\right)\Vec{u} \\
        &= \frac{\Vec{x_1}\cdot\Vec{u}}{\Vec{u}\cdot\Vec{u}}\Vec{u}+\frac{\Vec{x_2}\cdot\Vec{u}}{\Vec{u}\cdot\Vec{u}}\Vec{u} \\
        &= \func{T}{\Vec{x_1}}+\func{T}{x_2} 
    \end{align*}
    
    And, 
    
    \begin{align*}
        \func{T}{c\Vec{x_1}} &= \frac{(c\Vec{x_1})\cdot\Vec{u}}{\Vec{u}\cdot\Vec{u}}\Vec{u} \\ 
        &= \frac{c(\Vec{x_1}\cdot\Vec{u})}{\Vec{u}\cdot\Vec{u}}\Vec{u} \\
        &= c\left(\frac{\Vec{x_1}\cdot\Vec{u}}{\Vec{u}\cdot\Vec{u}}\Vec{u}\right) \\
        &= c \func{T}{\Vec{x}}
    \end{align*}
    
    Therefore $\operatorname{T}$ is a linear transformation.
    \end{Solution}
    
\end{document}
