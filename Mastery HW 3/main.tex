\documentclass[12pt, letterpaper]{article}

\usepackage[utf8]{inputenc}
\usepackage{systeme}
\usepackage{amsmath}
\usepackage{amssymb}
\usepackage{enumitem}
\usepackage{amsfonts}
\usepackage{amsthm}
\usepackage{graphicx}
\usepackage[colorinlistoftodos]{todonotes}
\usepackage{pifont}
\usepackage{mdframed,color}
\usepackage[left=3cm, right=3cm, top=3cm]{geometry}
\newcommand{\Z}{\mathbb{Z}}
\newcommand{\N}{\mathbb{N}}
\newcommand{\C}{\mathbb{C}}
\newcommand{\Q}{\mathbb{Q}}
\newcommand{\R}{\mathbb{R}}
\newcommand{\F}{\mathbb{F}}
\newtheoremstyle{statement}{3pt}{3pt}{}{}{\bfseries}{:}{.5em}{}

\theoremstyle{statement}
\newtheorem*{atmProp}{Proposition}

\theoremstyle{statement}
\newtheorem*{atmStat}{Statement}

\newenvironment{atmProof}{\noindent\ignorespaces\paragraph{Proof:}}{\hfill \ding{122}\par\noindent}

\newenvironment{Solution}{\noindent\ignorespaces\paragraph{Solution:}}{\hfill \ding{122}\par\noindent}

\newcount\arrowcount
\newcommand\arrows[1]{
        \global\arrowcount#1
        \ifnum\arrowcount>0
                \begin{matrix}
                \expandafter\nextarrow
        \fi
}

\newcommand\nextarrow[1]{
        \global\advance\arrowcount-1
        \ifx\relax#1\relax\else \xrightarrow{#1}\fi
        \ifnum\arrowcount=0
                \end{matrix}
        \else
                \\
                \expandafter\nextarrow
        \fi
}

\title{Mastery Homework 3}
\author{Rafael Laya}
\date{Fall 2018}

\begin{document}
    \maketitle

    \section*{Section 1.8}
        \subsection*{Problem 26.} 
        
        \begin{atmStat}
        Let $\Vec{u}$ and $\Vec{v}$ be linearly independent vectors in $\R^3$, and let $P$ be the plane through $\Vec{u}, \Vec{v}, $ and $\Vec{0}$. The parametric equation of $P$ is $\Vec{x} = s\Vec{u} + t\Vec{v}$ (with $s, t \in \R$). Show that a linear transformation $T: \R^3 \longrightarrow \R^3$ maps $P$ onto a plane through $\Vec{0}$, or onto a line through $\Vec{0}$, or onto just the origin in $\R^3$. What must be true about $\operatorname{T}(\Vec{u})$ and $\operatorname{T}(\Vec{v})$ in order for the image of the plane $P$ to be a plane?
        \end{atmStat}
        
        \begin{Solution}
        let $T: \R^3 \longrightarrow \R^3$ be a linear transformation. We want to apply $T$ to the parametric equation of the plane $P$ in order to apply linearity and see what corresponds to the parametric equation that we obtain. From there we will be able to deduce what has to be true about $\operatorname{T}(\Vec{u})$ and $\operatorname{T}(\Vec{v})$ in order for the image of the plane $P$ to be a plane. Therefore using the linearity of $T$ consider:
        
        \begin{align*}
            \operatorname{T}(\Vec{x}) & = \operatorname{T}(s\Vec{u} + \operatorname{t\Vec{v}})  \\
            & = \operatorname{T}(s\Vec{u}) +
            t\operatorname{T}(t\Vec{v}) \\
            & = s\operatorname{T}(\Vec{u}) + t\operatorname{T}(\Vec{v})  \\ 
        \end{align*}
        
        Every $\Vec{x}$ gets mapped as a linear combination of $\operatorname{T}(\Vec{u}), \operatorname{T}(\Vec{v})$. Therefore the plane $P$ gets mapped into $\operatorname{Span}( \{ \operatorname{T}(\Vec{u}), \operatorname{T}(\Vec{v})\})$. Then:
        
        \begin{itemize}
            \item If $\operatorname{T}(\Vec{u}), \operatorname{T}(\Vec{v})$ are both the zero vector, then $\operatorname{Span}(\{ \operatorname{T}(\Vec{u}), \operatorname{T}(\Vec{v})\}) = \{ \Vec{0}\}$ and the plane $P$ gets mapped directly into the Origin, $\Vec{0}$. See that if we let $\operatorname{T}(\Vec{u}) = \operatorname{T}(\Vec{v}) = \Vec{0}$ then: 
            
            $$\operatorname{T}(\Vec{x})=s(\Vec{0})+t(\Vec{0})=\Vec{0}+\Vec{0}=\Vec{0}$$
            
            \item If $\operatorname{T}(\Vec{u}), \operatorname{T}(\Vec{v})$ are linearly dependent and they are not both zero (say $\operatorname{T}(\Vec{u}) \neq \Vec{0}$, without loss of generality), then either one vector is zero and the Span is the line that goes through the origin and the other vector (say $\operatorname{T}(\Vec{u})$), or one is a multiple $k\neq0$ of the other and $\operatorname{Span}(\{ \operatorname{T}(\Vec{u}), \operatorname{T}(\Vec{v})\}) = 
            \operatorname{Span}(\{ \operatorname{T}(\Vec{u}), k\operatorname{T}(\Vec{u})\}) = 
            \operatorname{Span}(\{ \operatorname{T}(\Vec{u})\})$ or simply by theorem since $\{ \operatorname{T}(\Vec{u})\}$ is a linearly independent set and $\{ \operatorname{T}(\Vec{u}), \operatorname{T}(\Vec{v})\}$ is linearly dependent (after we added only one vector to the set) they have the same Span, which is a line through the origin. See that if we let $\operatorname{T}(\Vec{v})=k\operatorname{T}(\Vec{u})$ 
            $$\operatorname{T}(\Vec{x})=s\operatorname{T}(\Vec{u})+tk\operatorname{T}(\Vec{v}) = (s+tk)\operatorname{T}(\Vec{u})$$
            or if we let $\operatorname{T}(\Vec{v}) = \Vec{0}$ while $\operatorname{T}(\Vec{u})\neq \Vec{0}$ then:
            $$\operatorname{T}(\Vec{x})=s\operatorname{T}(\Vec{u})+t(\Vec{0})=s\operatorname{T}(\Vec{u})+\Vec{0}=s\operatorname{T}(\Vec{u})$$
            In any case, this is the equation of a line through the origin.
            \item If $\operatorname{T}(\Vec{u}), \operatorname{T}(\Vec{v})$ are linearly independent, then $\operatorname{Span}(\{ \operatorname{T}(\Vec{u}), \operatorname{T}(\Vec{v})\})$ is a Plane through the origin and through both vectors $\operatorname{T}(\Vec{u}), \operatorname{T}(\Vec{v})$ (the equation that led to this discussion is precisely the parametric equation of a plane).
        \end{itemize}
        
        Looking at what we have in our third bullet point we can also answer the last question: $\operatorname{T}(\Vec{u}), \operatorname{T}(\Vec{v})$ have to be linearly independent so that the image of the plane $P$ is mapped into another plane under the transformation $T$.
        
        \end{Solution}
    
        \subsection*{Problem 31}
        \begin{atmStat}
        Let $T: \R^n \longrightarrow \R^m$ be a linear transformation, and let $\{\Vec{v_1}, \Vec{v_2}, \Vec{v_3} \}$ be a linearly dependent set in $\R^n$. Explain Why the Set $\{ \operatorname{T}(\Vec{v_1}), \operatorname{T}(\Vec{v_2}), \operatorname{T}(\Vec{v_3})\}$ is linearly dependent.
     
        \end{atmStat}
        \begin{Solution}
        We will use the definition of linearly dependent set of vectors for $\{\Vec{v_1}, \Vec{v_2}, \Vec{v_3} \}$ and then we will apply the transformation $T$ to our resulting equation, by linearity we will be able to extract the coefficients of the linear combination and decide about the linearly dependency of the set $\{ \operatorname{T}(\Vec{v_1}), \operatorname{T}(\Vec{v_2}), \operatorname{T}(\Vec{v_3})\}$ (which will show atleast a non-trivial solution). Suppose then $\{\Vec{v_1}, \Vec{v_2}, \Vec{v_3} \}$ is a linearly dependent set in $\R^n$. That is, 
        
        $$x_1\Vec{v_1} + x_2\Vec{v_2} + x_3\Vec{v_3} = \Vec{0}$$
        
        Has a non-trivial solution $(c_1, c_2, c_3) \neq (0, 0, 0)$. Apply $T$ to the equation above when $(x_1, x_2, x_3) = (c_1, x_2, c_3)$
        \begin{align*}
            \operatorname{T}(c_1\Vec{v_1} + c_2\Vec{v_2} + c_3\Vec{v_3})
            & = \operatorname{T}(\Vec{0}) \\
        \end{align*}
        
        Using the linearity of $T$, and the fact that $\operatorname{T}(\Vec{0}) = \operatorname{T}(\Vec{0} + \Vec{0}) = \operatorname{T}(\Vec{0}) + \operatorname{T}(\Vec{0})$ and therefore $\operatorname{T}(\Vec{0}) = \Vec{0}$ (keep in mind that the input is the vector zero in $\R^n$ and the output is the vector zero in $\R^m$),
        \begin{align*}
            \operatorname{T}(c_1\Vec{v_1} + c_2\Vec{v_2} + c_3\Vec{v_3})
            & = \operatorname{T}(c_1\Vec{v_1})
            + \operatorname{T}(c_2\Vec{v_2}) 
            + \operatorname{T}(c_3\Vec{v_3}) \\
            & = c_1 \operatorname{T}(\Vec{v_1})
            + c_2 \operatorname{T}(\Vec{v_2})
            + c_3 \operatorname{T}(\Vec{v_3}) = \Vec{0}
        \end{align*}
        
        Therefore,  
        
        $$x_1\operatorname{T}(\Vec{v_1})
        + x_2\operatorname{T}(\Vec{v_2})
        + x_3\operatorname{T}(\Vec{v_3})
        = \Vec{0}$$
        Has a non-trivial solution $(x_1, x_2, x_3) = (c_1, c_2, c_3)$ and the set $\{ \operatorname{T}(\Vec{v_1}), \operatorname{T}(\Vec{v_2}), \operatorname{T}(\Vec{v_3})\}$ is a linearly dependent set.
        
        \end{Solution}
        
        \subsection*{Problem 34}
        \begin{atmStat}
        Let $T: \R^n \longrightarrow \R^m$ be a linear transformation. Show that if $T$ maps two linearly independent vectors onto a linearly dependent set, then the equation $\operatorname{T}(\Vec{x})=\Vec{0}$ has a nontrivial solution. 
        \end{atmStat}
        \begin{Solution}
        Let $\{\Vec{v_1}, \Vec{v_2}\}$ be a linearly independent set in $\R^n$ and suppose $\{ \operatorname{T}(\Vec{v_1}), \operatorname{T}(\Vec{v_2})\}$ is a linearly dependent set. Then: 
        
        $$x_1\Vec{v_1} + x_2\Vec{v_2} = \Vec{0}$$ 
        only has trivial solution and,
        $$x_1 \operatorname{T}(\Vec{v_1})
        + x_2 \operatorname{T}(\Vec{v_2})
        = \Vec{0}$$
        has a non-trivial solution $(x_1, x_2) = (c_1, c_2) \neq (0, 0)$. Then using linearity of $T$, 
        \begin{align*}
            c_1 \operatorname{T}(\Vec{v_1})
            + c_2 \operatorname{T}(\Vec{v_2})
            & = \operatorname{T}(c_1\Vec{v_1}) 
            + \operatorname{T}(c_2\Vec{v_2}) \\
            & = \operatorname{T}(c_1\Vec{v_1} + c_2\Vec{v_2})
        \end{align*}
        
        And so, 
        
        $$
        \operatorname{T}(c_1\Vec{v_1} + c_2\Vec{v_2})
        = \Vec{0}
        $$
        
        $\operatorname{T}(\Vec{x})=\Vec{0}$ has a non-trivial solution since the vector $\Vec{x} = c_1\Vec{v_1} + c_2\Vec{v_2}$ is a solution to $\operatorname{T}(\Vec{x})=\Vec{0}$ and it is not the zero vector since by assumption $x_1\Vec{v_1} + x_2\Vec{v_2}=\Vec{0}$ only has trivial solution but here $(c_1, c_2) \neq (0, 0)$.
        
        \end{Solution}
        
        \subsection*{Problem 36}
        \begin{atmStat}
        Let $T: \R^3 \longrightarrow \R^3$ be the transformation that projects each vector $\Vec{x} = (x_1, x_2, x_3)$ onto the plane $x_2 = 0$, So $\operatorname{T}(\Vec{x}) = (x_1,0, x_3)$. Show that $T$ is a linear transformation. 
        \end{atmStat}
        \begin{Solution}
        Let $\Vec{v} = (v_1, v_2, v_3)$ and $\Vec{u} = (u_1, u_2, u_3)$ be any two vectors in $\R^3$ and let $\lambda \in \R$. Now consider:
        
        \begin{align*}
            \operatorname{T}(\Vec{v}) +
            \operatorname{T}(\Vec{u})
            & = \operatorname{T}((v_1, v_2, v_3) + (u_1, u_2, u_3)) \\
            & = \operatorname{T}(v_1 + u_1, v_2 + u_2, v_3 + u_3) \\
            & = (v_1 + u_1, 0, v_3 + u_3) \\
            & = (v_1, 0, v_3) + (u_1, 0, u_3) \\
            & = \operatorname{T}(\Vec{v}) + \operatorname{T}(\Vec{u})
        \end{align*}
        And,
        \begin{align*}
            \operatorname{T}(\lambda\Vec{v})
            & = \operatorname{T}(\lambda(v_1, v_2, v_3)) \\
            & = \operatorname{T}(\lambda v_1, \lambda v_2, \lambda v_3) \\
            & = (\lambda v_1, 0, \lambda v_3) \\
            & = \lambda (v_1, 0, v_3) \\
            & = \lambda \operatorname{T}(\Vec{v})
        \end{align*}
        
        Since this is true for all $\Vec{v}, \Vec{u} \in \R^3$ and $\lambda \in \R$ Then $T$ is a linear transformation by definition.
        
        \end{Solution}
        
        \section*{1.9}
        
        \subsection*{Problem 22}
        \begin{atmStat}
        Let $T: \R^2 \longrightarrow \R^3$ be a linear transformation such that $\operatorname{T}(x_1, x_2) = (x_1-2x_2, -x_1+3x_2, 3x_1-2x_2)$ find $\Vec{x}$ such that $\operatorname{T}(\Vec{x})=(-1, 4, 9)$
        \end{atmStat}
        \begin{Solution}
        Given the formula of the linear transformation we can find $\operatorname{T}(\Vec{e_1})=\operatorname{T}(1,0)$ and $\operatorname{T}(\Vec{e_2})=\operatorname{T}(0, 1)$. By theorem 10 we can write the transformation matrix $A$ and then we will solve the system $A\Vec{x}=\Vec{b}$ where $\Vec{b} = \begin{bmatrix} -1 \\ 4 \\ 9\end{bmatrix}$. Let's find the images of $\Vec{e_1}, \Vec{e_2}$ under $T$,
        
        $$\operatorname{T}(\Vec{e_1})
        = \operatorname{T}(1, 0)
        = (1 - 2(0), -1 + 3(0), 3(1) - 2(0))
        = (1, -1, 3)$$
        $$\operatorname{T}(\Vec{e_2})
        = \operatorname{T}(0, 1)
        = (0 - 2(1), -0 + 3(1), 3(0) - 2(1))
        = (-2, 3, -2)$$
        
        
        Therefore the transformation matrix A is:
        
        $$ 
        A =
        \begin{bmatrix}
        1 & -2 \\
        -1 & 3 \\
        3 & 2
        \end{bmatrix}
        $$
        
        We can rewrite $\operatorname{T}(\Vec{x})$  as: $\operatorname{T}(\Vec{x}) = A\Vec{x}$. We want to find $\Vec{x}$ such that $\operatorname{T}(\Vec{x}) = \Vec{b}$ which is solved by row reducing the augmented matrix $\begin{bmatrix} A && \Vec{b}\end{bmatrix}$
        
        $$ 
        \begin{bmatrix}
        1 & -2 && -1\\
        -1 & 3 && 4\\
        3 & -2 && 9
        \end{bmatrix}
        \arrows3{}{R_2+R_1}{R_3-3R_1}
        \begin{bmatrix}
        1 & -2 && -1\\
        0 & 1 && 3\\
        0 & 4 && 12
        \end{bmatrix}
        \arrows3{R_1+2R_2}{}{R_3-4R_2}
        \begin{bmatrix}
        1 & 0 && 5\\
        0 & 1 && 3\\
        0 & 0 && 0
        \end{bmatrix}
        $$
        
        The solution is\\

        \systeme{
        x_1 = 5,
        x_2 = 3
        }\\
        
        In the desired form, $\Vec{x} = (5, 3)$ is such that $\operatorname{T}(\Vec{x}) = (-1, 4, 9)$
    
        \end{Solution}
    
\end{document}
