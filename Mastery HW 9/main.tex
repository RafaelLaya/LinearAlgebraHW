\documentclass[12pt, letterpaper]{article}

\usepackage[utf8]{inputenc}
\usepackage{float}
\usepackage{systeme}
\usepackage{amsmath}
\usepackage{amssymb}
\usepackage{enumitem}
\usepackage{amsfonts}
\usepackage{amsthm}
\usepackage{graphicx}
\usepackage[colorinlistoftodos]{todonotes}
\usepackage{pifont}
\usepackage{mdframed,color}
\usepackage[letterpaper, left=3cm, right=3cm, top=3cm, bottom=3cm]{geometry}
\usepackage{commath}
\newcommand{\Z}{\mathbb{Z}}
\newcommand{\N}{\mathbb{N}}
\newcommand{\C}{\mathbb{C}}
\newcommand{\Q}{\mathbb{Q}}
\newcommand{\R}{\mathbb{R}}
\newcommand{\F}{\mathbb{F}}
\newtheoremstyle{statement}{3pt}{3pt}{}{}{\bfseries}{:}{.5em}{}

\theoremstyle{statement}
\newtheorem*{atmProp}{Proposition}

\theoremstyle{statement}
\newtheorem*{atmStat}{Statement}

\newenvironment{atmProof}{\noindent\ignorespaces\paragraph{Proof:}}{\hfill \ding{122}\par\noindent}

\newenvironment{Solution}{\noindent\ignorespaces\paragraph{Solution:}}{\hfill \ding{122}\par\noindent}

\newcount\arrowcount
\newcommand\arrows[1]{
        \global\arrowcount#1
        \ifnum\arrowcount>0
                \begin{matrix}
                \expandafter\nextarrow
        \fi
}

\newcommand\nextarrow[1]{
        \global\advance\arrowcount-1
        \ifx\relax#1\relax\else \xrightarrow{#1}\fi
        \ifnum\arrowcount=0
                \end{matrix}
        \else
                \\
                \expandafter\nextarrow
        \fi
}

\newcommand{\func}[2]{\operatorname{#1}(#2)}
\newcommand{\trans}[2]{\func{#1}{\Vec{#2}}}

\title{Mastery Homework 9}
\author{Rafael Laya}
\date{Fall 2018}

\begin{document}
    \maketitle

    \section*{Section 6.3}
    \subsection*{Problem 24}
    \begin{atmStat}
    Let $W$ be a subspace of $\R^n$ with an orthogonal basis $\{\Vec{w_1}, \dots, \Vec{w_p}\}$, and let $\{\Vec{v_1}, \dots, \Vec{v_q}\}$ be an orthogonal basis for $W^\perp$. 
    
    a. Explain why $\{\Vec{w_1},\dots,\Vec{w_p},\Vec{v_1},\dots,\Vec{v_q}\}$ is an orthogonal set. 
    
    b. Explain why the set in part (a) spans $\R^n$.
    
    c. Show that $dim(W)+din(W^\perp)=n$.
    \end{atmStat}
    \begin{Solution}
    
    a. Recall the definition of $W^\perp$: 
    $$
    W^\perp = \left\{ \Vec{v} \in \R^n \middle\mid \Vec{v}\cdot\Vec{w}=0, \forall \Vec{w}\in W \right\}
    $$
    We know $\Vec{w_i}\cdot\Vec{w_j}=0$ for $i,j\in\{1,\dots,p\}$ when $i\neq j$ and $\Vec{v_i}\cdot\Vec{v_j}=0$ for $i,j\in\{1,\dots,q\}$ when $i\neq j$ by hypothesis. We only have to show that $\Vec{w_i}\cdot\Vec{v_j}$ for $i=1,\dots,p$ and $j=1,\dots, q$. Now, from the definition of $W^\perp$ we know that for any $\Vec{v}\in W$ and $\Vec{w} \in W^\perp$ we have $\Vec{v}\cdot\Vec{w}=0$. Notice then that $\Vec{w_i} \in W$ and $\Vec{v_j} \in W^\perp$ since they are part of a basis for $W$ and $W^\perp$, respectively.\\
    
    b. By the Orthogonal Decomposition Theorem, any vector $\Vec{x}$ can be written as $\Vec{x}=\Vec{x}_w+\Vec{x}_{w_\perp}$ where $\Vec{x_w}\in W$ and $\Vec{x}_{w_\perp} \in W^\perp$. Since $\{\Vec{w_1}, \dots, \Vec{w_p}\}$ is a basis for $W$ and $\{ \Vec{v_1}, \dots, \Vec{v_q}\}$ is a basis for $W^\perp$ there must exist scalars $\lambda_1, \dots, \lambda_p, \beta_1, \dots, \beta_q$ such that: 
    
    $$
    \lambda_1 \Vec{w}_1 + \dots + \lambda_p \Vec{w}_p = \Vec{x}_w
    $$
    And, 
    $$
    \beta_1 \Vec{v_1} + \dots + \lambda_q \Vec{v_q} = \Vec{x}_{w_\perp} 
    $$
    Adding these two equations:
    
    $$
    \lambda_1\Vec{w_1} + \dots + \lambda_p \Vec{w_p} + 
    \beta_1\Vec{v_1} + \dots + \beta_q \Vec{v_q}
    = \Vec{x}_w + \Vec{x}_{w_\perp}
    = \Vec{x}
    $$
    
    Therefore any $\Vec{x}\in\R^n$ is also in $\operatorname{Span}(\lambda_1\Vec{w_1}, \dots, \lambda_p\Vec{w_p}, \beta_1\Vec{v_1}, \dots, \beta_q \Vec{v_q})$, or simply $\operatorname{Span}(\lambda_1\Vec{w_1}, \dots, \lambda_p\Vec{w_p}, \beta_1\Vec{v_1}, \dots, \beta_q \Vec{v_q})=\R^n$ \\
    
    c. Let $S=\{\lambda_1\Vec{w_1}, \dots, \lambda_p\Vec{w_p}, \beta_1\Vec{v_1}, \dots, \beta_q \Vec{v_q}\}$We know that S spans $\R^n$ and that $S$ is an orthogonal set which by one of the first theorems of this chapter is a linearly independent set. This implies that $S$ is a basis for $\R^n$ and therefore it must contain $n$ vectors, $p+q=n$. Finally:
    $$
    \dim(\R^n)=\dim(S)=\dim(W)+\dim(W^\perp)=p+q=n
    $$
    Or simply:
    $$
    \dim(W)+\dim(W^\perp)=n
    $$
    Which is exactly what we wanted to show.
    \end{Solution}
    
    \section*{Section 6.5}
    \subsection*{Problem 14}
    \begin{atmStat}
    Let $A=\begin{bmatrix}
    2 & 1 \\
    -3 & -4 \\
    3 & 2
    \end{bmatrix}, \Vec{b} = \begin{bmatrix} 5\\4\\4\end{bmatrix}, \Vec{u}=\begin{bmatrix} 4\\-5\end{bmatrix},$ and $\Vec{v}=\begin{bmatrix}6\\-5\end{bmatrix}.$ Compute $A\Vec{u}$ and $A\Vec{v}$, and compare them with $\Vec{b}$. Is it possible that at least one of $\Vec{u}$ or $\Vec{v}$ could be a least-squares solution of $A\Vec{x}=\Vec{b}$?
    \end{atmStat}
    \begin{Solution}
    $$
        A\Vec{u} = 
        \begin{bmatrix} 
        2 & 1 \\
        -3 & -4 \\
        3 & 2
        \end{bmatrix} 
        \begin{bmatrix}
        4 \\ 
        -5
        \end{bmatrix} \\
        &=
        \begin{bmatrix}
        3 \\
        8 \\
        2
        \end{bmatrix} \\
    $$
    $$
        A\Vec{v} = 
        \begin{bmatrix} 
        2 & 1 \\
        -3 & -4 \\
        3 & 2
        \end{bmatrix} 
        \begin{bmatrix}
        6 \\ 
        -5
        \end{bmatrix} \\
        &=
        \begin{bmatrix}
        7 \\
        2 \\
        8
        \end{bmatrix} \\
    $$
    
    \begin{align*}
    \norm{\Vec{b}-A\Vec{u}} &= \norm{
    \begin{bmatrix}
    5 \\ 
    4 \\ 
    4
    \end{bmatrix}
    -
    \begin{bmatrix}
    3 \\ 
    8 \\ 
    2
    \end{bmatrix}
    } \\
    &=\norm{
    \begin{bmatrix}
    2 \\
    -4 \\
    2
    \end{bmatrix}
    } \\
    &= \sqrt{2^2 + 4^2 + 2^2} \\
    &= \sqrt{24}
    \end{align*}
    
    \begin{align*}
    \norm{\Vec{b}-A\Vec{v}} &= \norm{
    \begin{bmatrix}
    5 \\ 
    4 \\ 
    4
    \end{bmatrix}
    -
    \begin{bmatrix}
    7 \\ 
    2 \\ 
    8
    \end{bmatrix}
    } \\
    &=\norm{
    \begin{bmatrix}
    -2 \\
    2 \\
    -4
    \end{bmatrix}
    } \\
    &= \sqrt{2^2 + 2^2 + 4^2} \\
    &= \sqrt{24}
    \end{align*}
    
    By the definition alone of least-squares we can have multiple least-squares solutions. However, notice that:
    
    $$
    \det(A^T A) =
    \det\left(
    \begin{bmatrix} 
    2 & -3 & 3 \\ 
    1 & -4 & 2 
    \end{bmatrix}
    \begin{bmatrix}
    2 & 1\\
    -3 & -4\\
    3 & 2
    \end{bmatrix}
    \right)
    = 
    \det\left(
    \begin{bmatrix}
    22 & 20 \\
    20 &  21
    \end{bmatrix}
    \right)
    = 22(21)-20^2 = 62 \neq 0
    $$
    
    Therefore $A^TA$ is invertible and the least squares solution is unique. Therefore $\Vec{u}, \Vec{v}$ cannot be a least squares solution to $A\Vec{x}=\Vec{b}$.
    
    \end{Solution}
    
    \subsection*{Problem 25}
    \begin{atmStat}
    Describe all least-squares solutions of the system 
    
    \systeme{x+y=2,x+y=4}
    \end{atmStat}
    \begin{Solution}
    The system is equivalent to the matrix equation $A\Vec{x}=\Vec{b}$ where $A=\begin{bmatrix} 1 & 1 \\ 1 & 1 \end{bmatrix}$ and $\Vec{b}=\begin{bmatrix}2 \\ 4\end{bmatrix}$. Let's solve the normal system $A^TA=A^T\Vec{b}$: 
    
    $$
    A^TA = \begin{bmatrix} 
    1 & 1 \\
    1 & 1
    \end{bmatrix}
    \begin{bmatrix}
    1 & 1 \\
    1 & 1
    \end{bmatrix}
    = \begin{bmatrix}
    2 & 2 \\
    2 & 2
    \end{bmatrix}
    $$
    $$
    A^T\Vec{b} = 
    \begin{bmatrix}
    1 & 1 \\
    1 & 1 \\
    \end{bmatrix}
    \begin{bmatrix}
    2 \\ 4
    \end{bmatrix}
    =\begin{bmatrix}
    6 \\ 6
    \end{bmatrix}
    $$
    \end{Solution}
    
    Let's reduce the augmented matrix associated to the system:
    
    $$
    \begin{bmatrix}
    2 & 2 & 6 \\
    2 & 2 & 6 \\
    \end{bmatrix}
    \arrows2{}{R_2-R_1}
    \begin{bmatrix}
    2 & 2 & 6 \\
    0 & 0 & 0 \\
    \end{bmatrix}
    \arrows2{\frac{R_1}{2}}{}
    \begin{bmatrix}
    1 & 1 & 3 \\
    0 & 0 & 0 
    \end{bmatrix}
    $$
    The least squares solutions is any vector in the line
    $x+y=3$ which is simply a line in between $x+y=2$ and $x+y=4$.
    
    And the image of any least-squares solution is ($x=3-y$ from $x+y=3$): 
    
    $$
    A\Vec{x}=
    \begin{bmatrix}
    1 & 1 \\ 
    1 & 1
    \end{bmatrix}
    \begin{bmatrix}
    3 - y \\ 
    y
    \end{bmatrix}
    =
    \begin{bmatrix}
    3-y+y \\
    3-y+y
    \end{bmatrix}
    =\begin{bmatrix}
    3\\3
    \end{bmatrix}
    $$
    With error:
    $$
    \norm{\Vec{b}-A\Vec{x}} 
    =
    \norm{
    \begin{bmatrix} 
    2 \\
    4 
    \end{bmatrix}
    -
    \begin{bmatrix}
    3 \\
    3
    \end{bmatrix}
    }
    =
    \norm{
    \begin{bmatrix}
    -1 \\
    1
    \end{bmatrix}
    }
    =
    \sqrt{1^2+1^2}
    = \sqrt{2}
    $$
\end{document}
